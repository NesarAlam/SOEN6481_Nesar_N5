\hspace{4.5cm}   \underline{Problem 2}

\hspace{3.2cm}      Interviewee- Niloy Eric Costa \newline

He is a MSc candidate in Computer Science at York University. His supervisor is Manos Pappagelis and he is a member of the Data Mining Lab. He works in Interactive Data Visualization.\newline

Q1. What are you researching on?


A1. I am working on Interactive Data Visualization.\newline

Q2. Are there any other fields in Computer Science or Mathematics you are interested in?

A2. In the field of Mathematics I'm interested in Geometry and Number Systems and in the field of Computer Science I'm interested in Data Visualization, Database Systems and Data Mining.
\newline 


Q3. Have you heard of any irrational constants in Mathematics? 


A3. I have researched on some irrational constants like $\pi$,e,Khinchins Constant (K).
\newline

     Q4. What do you know about Golden Ratio,$\Phi$ ?
     
     
     A4. Well to explain it.Two quantities are in the golden ratio if their ratio is the same as the ratio of their sum to the
larger of the two quantities. For quantities a and b such that $a > b > 0$,
$$(a+b)/a =a/b$$
     \newline

    Q5. Do you know when we celebrate phi day?
    
    
    A5. Haha. Yeah, 18th June.
    \newline
    
    Q6. Phi is named after a Greek sculptor; do you know his name?
    
    
    A6. Yes, Phidias.
    \newline

   Q7. Do you know any other names golden ratio is known by?
   
   
   A7. Yes.It is known by the Golden Mean, Phi, the Divine Section, The Golden Cut, The Golden Proportion, The Divine Proportion, and tau(t).
   \newline
   
   Q8. How do you derive the value of golden ratio? 
   
   
   A8. By using the quadratic equation $x^2-x-1=0$
   \newline
   
   Q9. Can you tell us about any applications of Golden Ratio?
   
   
   A9. The golden ratio is used mostly in the Geometry to create designs that are in proportions and are
pleasing to the eye. It is not used as such in Mathematics directly but even the ratio of consecutive
numbers in fibonacci series are close to the golden ratio.
   \newline

   Q10. Can you describe some uses of Golden Ratio in architecture and art?
   
   
   A10. The Great Pyramids of Gaza, Parthenon in Athens, Michelangelo’s The Creation of Adam on the
ceiling of the Sistine Chapel and Da Vinci’s Mona Lisa are some of the famous examples that use the
Golden Ratio.
   \newline

  Q11. The ancient Egyptians used the golden ratio in their pyramids. At that time,     the golden was known to them by another name, do you know the name?
  
  
  A11. The Sacred Ratio.
  \newline

   Q12. Do you know about any other fields golden ratio is claimed to appear?
   
   
   A12. The golden ratio is claimed to appear in many fields, such as cosmology, theology, arts, architecture, botany and others.
   \newline
   
   Q14.Would you like to include Irrational constants like Golden ratio in the calculator?


A14. Yes. I would prefer it.

